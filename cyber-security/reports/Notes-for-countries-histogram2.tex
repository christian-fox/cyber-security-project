\documentclass[]{article}
\usepackage{lmodern}
\usepackage{amssymb,amsmath}
\usepackage{ifxetex,ifluatex}
\usepackage{fixltx2e} % provides \textsubscript
\ifnum 0\ifxetex 1\fi\ifluatex 1\fi=0 % if pdftex
  \usepackage[T1]{fontenc}
  \usepackage[utf8]{inputenc}
\else % if luatex or xelatex
  \ifxetex
    \usepackage{mathspec}
  \else
    \usepackage{fontspec}
  \fi
  \defaultfontfeatures{Ligatures=TeX,Scale=MatchLowercase}
\fi
% use upquote if available, for straight quotes in verbatim environments
\IfFileExists{upquote.sty}{\usepackage{upquote}}{}
% use microtype if available
\IfFileExists{microtype.sty}{%
\usepackage[]{microtype}
\UseMicrotypeSet[protrusion]{basicmath} % disable protrusion for tt fonts
}{}
\PassOptionsToPackage{hyphens}{url} % url is loaded by hyperref
\usepackage[unicode=true]{hyperref}
\hypersetup{
            pdftitle={Notes for countries histogram},
            pdfauthor={Christian Fox},
            pdfborder={0 0 0},
            breaklinks=true}
\urlstyle{same}  % don't use monospace font for urls
\usepackage[margin=1in]{geometry}
\usepackage{color}
\usepackage{fancyvrb}
\newcommand{\VerbBar}{|}
\newcommand{\VERB}{\Verb[commandchars=\\\{\}]}
\DefineVerbatimEnvironment{Highlighting}{Verbatim}{commandchars=\\\{\}}
% Add ',fontsize=\small' for more characters per line
\usepackage{framed}
\definecolor{shadecolor}{RGB}{248,248,248}
\newenvironment{Shaded}{\begin{snugshade}}{\end{snugshade}}
\newcommand{\KeywordTok}[1]{\textcolor[rgb]{0.13,0.29,0.53}{\textbf{#1}}}
\newcommand{\DataTypeTok}[1]{\textcolor[rgb]{0.13,0.29,0.53}{#1}}
\newcommand{\DecValTok}[1]{\textcolor[rgb]{0.00,0.00,0.81}{#1}}
\newcommand{\BaseNTok}[1]{\textcolor[rgb]{0.00,0.00,0.81}{#1}}
\newcommand{\FloatTok}[1]{\textcolor[rgb]{0.00,0.00,0.81}{#1}}
\newcommand{\ConstantTok}[1]{\textcolor[rgb]{0.00,0.00,0.00}{#1}}
\newcommand{\CharTok}[1]{\textcolor[rgb]{0.31,0.60,0.02}{#1}}
\newcommand{\SpecialCharTok}[1]{\textcolor[rgb]{0.00,0.00,0.00}{#1}}
\newcommand{\StringTok}[1]{\textcolor[rgb]{0.31,0.60,0.02}{#1}}
\newcommand{\VerbatimStringTok}[1]{\textcolor[rgb]{0.31,0.60,0.02}{#1}}
\newcommand{\SpecialStringTok}[1]{\textcolor[rgb]{0.31,0.60,0.02}{#1}}
\newcommand{\ImportTok}[1]{#1}
\newcommand{\CommentTok}[1]{\textcolor[rgb]{0.56,0.35,0.01}{\textit{#1}}}
\newcommand{\DocumentationTok}[1]{\textcolor[rgb]{0.56,0.35,0.01}{\textbf{\textit{#1}}}}
\newcommand{\AnnotationTok}[1]{\textcolor[rgb]{0.56,0.35,0.01}{\textbf{\textit{#1}}}}
\newcommand{\CommentVarTok}[1]{\textcolor[rgb]{0.56,0.35,0.01}{\textbf{\textit{#1}}}}
\newcommand{\OtherTok}[1]{\textcolor[rgb]{0.56,0.35,0.01}{#1}}
\newcommand{\FunctionTok}[1]{\textcolor[rgb]{0.00,0.00,0.00}{#1}}
\newcommand{\VariableTok}[1]{\textcolor[rgb]{0.00,0.00,0.00}{#1}}
\newcommand{\ControlFlowTok}[1]{\textcolor[rgb]{0.13,0.29,0.53}{\textbf{#1}}}
\newcommand{\OperatorTok}[1]{\textcolor[rgb]{0.81,0.36,0.00}{\textbf{#1}}}
\newcommand{\BuiltInTok}[1]{#1}
\newcommand{\ExtensionTok}[1]{#1}
\newcommand{\PreprocessorTok}[1]{\textcolor[rgb]{0.56,0.35,0.01}{\textit{#1}}}
\newcommand{\AttributeTok}[1]{\textcolor[rgb]{0.77,0.63,0.00}{#1}}
\newcommand{\RegionMarkerTok}[1]{#1}
\newcommand{\InformationTok}[1]{\textcolor[rgb]{0.56,0.35,0.01}{\textbf{\textit{#1}}}}
\newcommand{\WarningTok}[1]{\textcolor[rgb]{0.56,0.35,0.01}{\textbf{\textit{#1}}}}
\newcommand{\AlertTok}[1]{\textcolor[rgb]{0.94,0.16,0.16}{#1}}
\newcommand{\ErrorTok}[1]{\textcolor[rgb]{0.64,0.00,0.00}{\textbf{#1}}}
\newcommand{\NormalTok}[1]{#1}
\usepackage{graphicx,grffile}
\makeatletter
\def\maxwidth{\ifdim\Gin@nat@width>\linewidth\linewidth\else\Gin@nat@width\fi}
\def\maxheight{\ifdim\Gin@nat@height>\textheight\textheight\else\Gin@nat@height\fi}
\makeatother
% Scale images if necessary, so that they will not overflow the page
% margins by default, and it is still possible to overwrite the defaults
% using explicit options in \includegraphics[width, height, ...]{}
\setkeys{Gin}{width=\maxwidth,height=\maxheight,keepaspectratio}
\IfFileExists{parskip.sty}{%
\usepackage{parskip}
}{% else
\setlength{\parindent}{0pt}
\setlength{\parskip}{6pt plus 2pt minus 1pt}
}
\setlength{\emergencystretch}{3em}  % prevent overfull lines
\providecommand{\tightlist}{%
  \setlength{\itemsep}{0pt}\setlength{\parskip}{0pt}}
\setcounter{secnumdepth}{0}
% Redefines (sub)paragraphs to behave more like sections
\ifx\paragraph\undefined\else
\let\oldparagraph\paragraph
\renewcommand{\paragraph}[1]{\oldparagraph{#1}\mbox{}}
\fi
\ifx\subparagraph\undefined\else
\let\oldsubparagraph\subparagraph
\renewcommand{\subparagraph}[1]{\oldsubparagraph{#1}\mbox{}}
\fi

% set default figure placement to htbp
\makeatletter
\def\fps@figure{htbp}
\makeatother


\title{Notes for countries histogram}
\author{Christian Fox}
\date{28 November 2020}

\begin{document}
\maketitle

\subsection{Business Understanding}\label{business-understanding}

Project objectives were come up with, rather than have the
business/customer give requirements and objectives. Data about the
students country of residence was chosen to extract. This data can be
used on its own to determine how much foreign attention the course
attracts, which allows the business to target their advertising as wells
as taylor/specify the course to these countries cultures and current
events in these societies which would enable students to take away the
most from the course.

Since there is no client in this case, this phase is almost obselete for
this certain project. Having data mining being done within a business
indedendently proves a downfall for the CRISP-DM methodology.

\subsection{Data Understanding}\label{data-understanding}

Data collection is usually done after the requirements and objectives
are given, however in this case the data is already given. Becoming
familiar with the data involved finding; \emph{total amount of students
}total unenrolments \emph{total course completion }total undetected
countries + which turns out to be a data quality problem.

Some of this data manipluation can be seen here:

\begin{Shaded}
\begin{Highlighting}[]
\KeywordTok{sum}\NormalTok{(cse}\OperatorTok{$}\NormalTok{fully_participated_at }\OperatorTok{!=}\StringTok{ ""}\NormalTok{)   }\CommentTok{# amount of students that did NOT fully participate}
\KeywordTok{sum}\NormalTok{(cse}\OperatorTok{$}\NormalTok{unenrolled_at }\OperatorTok{!=}\StringTok{ ""}\NormalTok{)           }\CommentTok{# amount of students that did NOT unenroll}
\KeywordTok{message}\NormalTok{(}\StringTok{"the total number of students = "}\NormalTok{, }\KeywordTok{sum}\NormalTok{(count.df}\OperatorTok{$}\NormalTok{Freq))}
\KeywordTok{message}\NormalTok{(}\StringTok{"1% of students = "}\NormalTok{, }\KeywordTok{sum}\NormalTok{(count.df}\OperatorTok{$}\NormalTok{Freq)}\OperatorTok{/}\DecValTok{100}\NormalTok{)}
\KeywordTok{message}\NormalTok{(}\StringTok{"number of students with no detected country = "}\NormalTok{, count.df[count.df}\OperatorTok{$}\NormalTok{list_of_countries}\OperatorTok{==}\StringTok{"--"}\NormalTok{,]}\OperatorTok{$}\NormalTok{Freq)}
\KeywordTok{message}\NormalTok{(}\StringTok{"% of students with no detected country = "}\NormalTok{, count.df[count.df}\OperatorTok{$}\NormalTok{list_of_countries}\OperatorTok{==}\StringTok{"--"}\NormalTok{,]}\OperatorTok{$}\NormalTok{Freq}\OperatorTok{*}\DecValTok{100} \OperatorTok{/}\KeywordTok{sum}\NormalTok{(count.df}\OperatorTok{$}\NormalTok{Freq))}
\end{Highlighting}
\end{Shaded}

Data understanding can also involve detecting interseting data subsets
to form hypothesis regarding hidden information, for example, identify
the students whos country is unknown, look at their total watch-time of
the lectures and compare theirs to the mean average of students from
Great Britian, which we can assume most of these students will be
attending lectures in person, rather than remotely. This enables us to
make an estimation on whether the student is studying remotely, and
hence residing in a foreign country.

I decided to use data for the students detected country.

\begin{Shaded}
\begin{Highlighting}[]
\CommentTok{# could make column width smaller here.}
\KeywordTok{head}\NormalTok{(cse) }\CommentTok{# showing first 6 rows}
\end{Highlighting}
\end{Shaded}

\begin{verbatim}
##                             learner_id             enrolled_at
## 1 160d6600-ea0e-4568-bfa9-5d7cd5b8e61b 2016-08-10 14:28:49 UTC
## 2 4dc22fed-63d4-4bf6-b162-bdf482e1ec38 2016-05-24 17:34:34 UTC
## 3 ecdd37db-0c75-496e-bff2-230553d0e38c 2016-05-19 00:52:38 UTC
## 4 988964c9-7410-40cc-addf-441f93e7a8b8 2016-05-19 21:40:01 UTC
## 5 f1493366-17a1-41b8-9de3-fbaad9d811d4 2016-09-19 15:35:35 UTC
## 6 25cc3b46-a955-4e2a-a71f-6b2025cc2787 2016-08-30 04:16:43 UTC
##             unenrolled_at    role   fully_participated_at
## 1                         learner                        
## 2 2018-10-30 20:20:51 UTC learner                        
## 3                         learner 2016-09-22 16:56:03 UTC
## 4                         learner                        
## 5                         learner                        
## 6                         learner 2016-10-25 12:44:14 UTC
##   purchased_statement_at  gender country age_range highest_education_level
## 1                        Unknown Unknown   Unknown                 Unknown
## 2                           male      PE     46-55       university_degree
## 3                        Unknown Unknown   Unknown                 Unknown
## 4                        Unknown Unknown   Unknown                 Unknown
## 5                        Unknown Unknown   Unknown                 Unknown
## 6                        Unknown Unknown   Unknown                 Unknown
##   employment_status        employment_area detected_country
## 1           Unknown                Unknown               GB
## 2 working_part_time teaching_and_education               PE
## 3           Unknown                Unknown               NG
## 4           Unknown                Unknown               UG
## 5           Unknown                Unknown               IM
## 6           Unknown                Unknown               NO
\end{verbatim}

\subsection{Notes on data:}\label{notes-on-data}

To be able to easily manipulate the detected\_country column of this
data I named the variable and set this coulmn as a data frame.

\begin{Shaded}
\begin{Highlighting}[]
\CommentTok{# first create data frame with countries and frequency:}
\NormalTok{list_of_countries =}\StringTok{ }\NormalTok{cse}\OperatorTok{$}\NormalTok{detected_country}
\CommentTok{# now find the 'count'/frequency of each country}
\NormalTok{count.df =}\StringTok{ }\KeywordTok{as.data.frame}\NormalTok{(}\KeywordTok{table}\NormalTok{(list_of_countries))}
\end{Highlighting}
\end{Shaded}

\subsubsection{CRISP-DM methodology
update}\label{crisp-dm-methodology-update}

A further undetsranding of the data has been achieved by munging the
data. Therefore, I was subconsciously alternating from data
understanding to data preperation.

The total number of students is 14383. There are 428 students who's
country is unknown. This gives a percentage of students with no detected
country of 2.9757352\%.

A histogram for countries was plotted:
\includegraphics{Notes-for-countries-histogram2_files/figure-latex/unnamed-chunk-2-1.pdf}

\subsubsection{CRISP-DM methodology
update}\label{crisp-dm-methodology-update-1}

Producing a histogram was the first modelling step. After a first look
at the graph, it is clear that a back-track in the steps in necessary,
since the x labels are unreadable.

\subsubsection{Data understanding}\label{data-understanding-1}

Studying the data frame of countries, it is seen that there are 183
unique countries with a large portion only having 1 or 2 students.

\subsubsection{Back to data preperation}\label{back-to-data-preperation}

This lead to sorting through the countries list and a decision was made
to omit countries with under 100 participating students, as well as the
undetected country row.

\begin{Shaded}
\begin{Highlighting}[]
\CommentTok{# omitting the countries with < 100 students/frequency}
\NormalTok{countries_over_}\DecValTok{100}\NormalTok{ =}\StringTok{ }\NormalTok{count.df[}\KeywordTok{which}\NormalTok{(count.df}\OperatorTok{$}\NormalTok{Freq }\OperatorTok{>=}\StringTok{ }\DecValTok{100} \OperatorTok{&}\StringTok{ }\NormalTok{count.df}\OperatorTok{$}\NormalTok{list_of_countries }\OperatorTok{!=}\StringTok{ "--"}\NormalTok{),]}\OperatorTok{$}\NormalTok{list_of_countries}
\end{Highlighting}
\end{Shaded}

The following hosogram was produced:

\includegraphics{Notes-for-countries-histogram2_files/figure-latex/unnamed-chunk-4-1.pdf}

This plot is more readable, as well as more relevant as the undetected
countries bar didnt provide much useful information.

\subsection{Running the source code with the data for different
years}\label{running-the-source-code-with-the-data-for-different-years}

There were 7 different cohorts with data for the students' detected
country. To make the code more simple to change between the different
cohorts, the munge file was edited to have an extra variable:

\begin{Shaded}
\begin{Highlighting}[]
\NormalTok{cohort =}\StringTok{ }\DecValTok{1}    \CommentTok{# this is the different years of data, ranging from 1-7.}
\NormalTok{cse =}\StringTok{ }\KeywordTok{data.frame}\NormalTok{(}\KeywordTok{read.table}\NormalTok{(}\KeywordTok{paste}\NormalTok{(}\StringTok{'cyber-security-'}\NormalTok{,cohort,}\StringTok{'_enrolments.csv'}\NormalTok{, }\DataTypeTok{sep=}\StringTok{""}\NormalTok{), }\DataTypeTok{header=}\OtherTok{TRUE}\NormalTok{, }\DataTypeTok{sep=}\StringTok{','}\NormalTok{))}
\end{Highlighting}
\end{Shaded}

The code was then manually changed in ascending order until cohort 3 was
ran. This cohort produced a histogram with only 3 countries with over
100 students. This caused a re-evaluation, and the CRISP-DM phase to
back-track to the data understanding once again.

It was clear that this model had a problem with producing the final
histogram as there is no lower limit to the amount of students enrolling
on the course each year. A solution was manifested. The arbitrary and
non-conformative/non-comprehensive/non-flexible number of 100 was
changed to a percentage, which doesnt fully solve the problem of gaining
an empty histogram some years (although it does decrease the chances of
it happening dramatically), of 1\%. There is definitely more
sophisticated models to let the histogram be consistent eaxh year
available, such as choosing the top 10 countries etc. however we will
not be assessed on either coming up with the mathematical model or
coding the better solution.

Running each cohort with the 1\% condition, consistent, readable
histograms are created for every year.

\subsubsection{Endless evaluation of the data extraction process could
be done, hovever the pipeline will have to eventually be deployed, as
the model will expose its flaws best in an environment most specific to
where it will be
used.}\label{endless-evaluation-of-the-data-extraction-process-could-be-done-hovever-the-pipeline-will-have-to-eventually-be-deployed-as-the-model-will-expose-its-flaws-best-in-an-environment-most-specific-to-where-it-will-be-used.}

\end{document}
